%-----------------------------------------------------------------
%	BASIC DOCUMENT LAYOUT
%-----------------------------------------------------------------
\documentclass[paper=a4, fontsize=12pt, twoside=semi]{scrartcl}
\usepackage[T1]{fontenc}
\usepackage[utf8]{inputenc}
\usepackage{lmodern}
\usepackage{slantsc}
\usepackage{microtype}
\usepackage[british]{babel}
\usepackage[backend=bibtex, style=trad-abbrv, sorting=none, maxbibnames=3, maxcitenames=2]{biblatex}
\addbibresource{bibliography.bib}
\makeatletter
	\def\blx@maxline{77}
\makeatother


% Sectioning layout
\addtokomafont{sectioning}{\normalfont\scshape}
\usepackage{tocstyle}
\usetocstyle{standard}
\renewcommand*\descriptionlabel[1]{\hspace\labelsep\normalfont\bfseries{#1}}

% Empty pages
\usepackage{etoolbox}

% Paragraph indentation behaviour
\setlength{\parindent}{0pt}
\setlength{\parskip}{0.3\baselineskip plus2pt minus2pt}
\newcommand{\sk}{\medskip\noindent}

% Fancy header and footer
\usepackage{fancyhdr}
\pagestyle{fancyplain}
\fancyhead[L]{\mytitle}
\fancyhead[R]{\myshortauthor}
\fancyfoot[C]{\thepage}
\fancypagestyle{firststyle}
{
	\fancyhead[L]{\myauthor}
	\fancyhead[R]{\mydate}
	\fancyfoot[C]{\thepage}
}
\renewcommand{\headrulewidth}{0.3pt}
\renewcommand{\footrulewidth}{0pt}
\setlength{\headheight}{13.6pt}

%-----------------------------------------------------------------
%	MATHS AND SCIENCE
%-----------------------------------------------------------------
\usepackage{amsmath,amsfonts,amsthm,amssymb}
\usepackage{xfrac}
\usepackage[a]{esvect}
\usepackage{chemformula}
\usepackage{graphicx}

\usepackage[arrowdel]{physics}
	\renewcommand{\vnabla}{\vec{\nabla}}
	% \renewcommand{\vectorbold}[1]{\boldsymbol{#1}}
	% \renewcommand{\vectorarrow}[1]{\vec{\boldsymbol{#1}}}
	% \renewcommand{\vectorunit}[1]{\hat{\boldsymbol{#1}}}
	\renewcommand{\vectorarrow}[1]{\vec{#1}}
	\renewcommand{\vectorunit}[1]{\hat{#1}}
	\renewcommand*{\grad}[1]{\vnabla #1}
	\renewcommand*{\div}[1]{\vnabla \vdot \va{#1}}
	\renewcommand*{\curl}[1]{\vnabla \cp \va{#1}}
	\let\rot\curl

% SI units
\usepackage[separate-uncertainty=true]{siunitx}
\sisetup{range-phrase = \text{--}, range-units = brackets}
\DeclareSIPrePower\quartic{4}
	%\DeclareSIUnit\micron{\micro\metre}

% Smaller trig functions
\newcommand{\Sin}{\trigbraces{\operatorname{s}}}
\newcommand{\Cos}{\trigbraces{\operatorname{c}}}
\newcommand{\Tan}{\trigbraces{\operatorname{t}}}

% Operator-style notation for matrices
\newcommand*{\mat}[1]{\hat{#1}}

% Matrices in (A|B) form via [c|c] option
\makeatletter
\renewcommand*\env@matrix[1][*\c@MaxMatrixCols c]{%
	\hskip -\arraycolsep
	\let\@ifnextchar\new@ifnextchar
	\array{#1}}
\makeatother

% Shorter \mathcal and \mathbb
\newcommand*{\mc}[1]{\mathcal{#1}}
\newcommand*{\mbb}[1]{\mathbb{#1}}

% Shorter ^\ast and ^\dagger
\newcommand*{\sast}{^{\star}{}}
\newcommand*{\sdag}{^{\dagger}{}}

% Complex and Hermitian conjugates
\newcommand*{\cc}{\,\text{c.c.}}
\newcommand*{\Hc}{\,\text{H.c.}}

% Blackboard bold identity
\usepackage{bbm}
\newcommand*{\bbid}{\mathbbm{1}}

% Shorter displaystyle
\newcommand*{\dsp}{\displaystyle}

%-----------------------------------------------------------------
%	OTHER PACKAGES
%-----------------------------------------------------------------
\usepackage{environ}

% Plots and graphics
\usepackage{pgfplots}
\usepackage{tikz}
	\usetikzlibrary{calc}
\usepackage{color}
	\makeatletter
		\color{black}
		\let\default@color\current@color
	\makeatother

% Richer enumerate, figure, and table support
\usepackage{enumerate}
\usepackage[shortlabels]{enumitem}
\usepackage{float}
\usepackage{tabularx}
\usepackage{booktabs}
	%\setlength{\intextsep}{8pt}
\numberwithin{equation}{section}
\numberwithin{figure}{section}
\numberwithin{table}{section}

% No indentation after certain environments
\makeatletter
\newcommand*\NoIndentAfterEnv[1]{%
	\AfterEndEnvironment{#1}{\par\@afterindentfalse\@afterheading}}
\makeatother
% \NoIndentAfterEnv{thm}
\NoIndentAfterEnv{defi}
\NoIndentAfterEnv{example}
\NoIndentAfterEnv{table}

% Misc packages
\usepackage{ccicons}
\usepackage{lipsum}
\usepackage{todonotes}
\usepackage{array}
\usepackage{multirow}

% Print DOI only if there's no URL
% \renewbibmacro*{doi+eprint+url}{%
%   \iftoggle{bbx:doi}
%     {\iffieldundef{url}{\printfield{doi}}{}}
%     {}%
%   \newunit\newblock
%   \iftoggle{bbx:eprint}
%     {\usebibmacro{eprint}}
%     {}%
%   \newunit\newblock
%   \iftoggle{bbx:url}
%     {\usebibmacro{url+urldate}}
%     {}}

%-----------------------------------------------------------------
%	SYNTAX HIGHLIGHTING
%-----------------------------------------------------------------
\usepackage[formats]{listings}
\usepackage{relsize}
\usepackage{chngcntr}

\renewcommand{\lstlistingname}{Snippet}
\renewcommand{\lstlistlistingname}{List of snippets}

\lstloadlanguages{R,bash,python,sql}
\lstdefinelanguage{Renhanced}[]{R}{%
	morekeywords={acf,ar,arima,arima.sim,colMeans,colSums,is.na,is.null,%
	mapply,ms,na.rm,nlmin,replicate,row.names,rowMeans,rowSums,seasonal,%
	sys.time,system.time,ts.plot,which.max,which.min,%
	rename,mutate,unite,select,filter,left_join,group_by,dplyr::select,%
	ggplot,aes,geom_line,geom_hline,geom_point,geom_path,geom_errorbar,%
	geom_abline,geom_smooth%
	geom_cartogram,coord_proj,scale_x_longitude, scale_y_latitude,%
	labs,guides,annotate,theme,rowwise,%
	scale_linetype_manual,scale_colour_manual,scale_x_log10,scale_y_log10,%
	attr,paste,paste0,bind_rows,str_trim,as.numeric,as.dataframe,data.frame},
	deletekeywords={c,range,step},
	alsoletter={.,_,::},
	otherkeywords = {!,!=,~,\$,*,\&,\%/\%,\%*\%,\%\%,\%>\%,<-,<<-,\% in \%}
	}


\newcommand*{\inline}{\lstinline[basicstyle=\normalsize\ttfamily]}
\newcommand*{\inlinev}{\lstinline[basicstyle=\normalsize\ttfamily, showtabs=true, showspaces=true]}

\lstset{language=sql,
		frame=tb,
		% captionpos=b,
		tabsize=2,
		% showtabs=true,
		breaklines=true,
		breakatwhitespace=true,
		basicstyle=\smaller\ttfamily,
		numbers=left,
		numberstyle=\tiny,
		numbersep=7.5pt,
		% commentstyle=\textsl,
		xleftmargin=3ex}
% \lstset{escapeinside={(*}{*)}}   % for (*\ref{ }*) inside lstlistings (Scode)

\lstdefinestyle{output}{
	language=,
	% showtabs=true,
	% showspaces=true,
	numbers=none,
	frame=tblr,
	% columns=fullflexible,
	backgroundcolor=\color{blue!10},
	% numbers=left,
	% numberstyle=\tiny,
	% numbersep=7.5pt,
	xleftmargin=3ex}

%-----------------------------------------------------------------
%	THEOREMS
%-----------------------------------------------------------------
\usepackage{thmtools}

% Theroems layout
\declaretheoremstyle[
	spaceabove=6pt, spacebelow=6pt,
	headfont=\normalfont,
	notefont=\mdseries, notebraces={(}{)},
	bodyfont=\small,
	postheadspace=1em,
]{small}

\declaretheorem[style=plain,name=Theorem,qed=$\square$,numberwithin=section]{thm}
\declaretheorem[style=plain,name=Corollary,qed=$\square$,sibling=thm]{cor}
\declaretheorem[style=plain,name=Lemma,qed=$\square$,sibling=thm]{lem}
\declaretheorem[style=definition,name=Definition,qed=$\blacksquare$,numberwithin=section]{defi}
\declaretheorem[style=definition,name=Example,qed=$\blacktriangle$,numberwithin=section]{example}
\declaretheorem[style=small,name=Proof,numbered=no,qed=$\square$]{sproof}
\declaretheorem[style=definition,name=Answer]{answ}

%-----------------------------------------------------------------
%	ELA MOTHERFUCKING GEMINADA
%-----------------------------------------------------------------
\def\xgem{%
	\ifmmode
		\csname normal@char\string"\endcsname l%
	\else
		\leftllkern=0pt\rightllkern=0pt\raiselldim=0pt
		\setbox0\hbox{l}\setbox1\hbox{l\/}\setbox2\hbox{.}%
		\advance\raiselldim by \the\fontdimen5\the\font
		\advance\raiselldim by -\ht2
		\leftllkern=-.25\wd0%
		\advance\leftllkern by \wd1
		\advance\leftllkern by -\wd0
		\rightllkern=-.25\wd0%
		\advance\rightllkern by -\wd1
		\advance\rightllkern by \wd0
		\allowhyphens\discretionary{-}{}%
		{\kern\leftllkern\raise\raiselldim\hbox{.}%
			\kern\rightllkern}\allowhyphens
	\fi
}
\def\Xgem{%
	\ifmmode
		\csname normal@char\string"\endcsname L%
	\else
		\leftllkern=0pt\rightllkern=0pt\raiselldim=0pt
		\setbox0\hbox{L}\setbox1\hbox{L\/}\setbox2\hbox{.}%
		\advance\raiselldim by .5\ht0
		\advance\raiselldim by -.5\ht2
		\leftllkern=-.125\wd0%
		\advance\leftllkern by \wd1
		\advance\leftllkern by -\wd0
		\rightllkern=-\wd0%
		\divide\rightllkern by 6
		\advance\rightllkern by -\wd1
		\advance\rightllkern by \wd0
		\allowhyphens\discretionary{-}{}%
		{\kern\leftllkern\raise\raiselldim\hbox{.}%
			\kern\rightllkern}\allowhyphens
	\fi
}

\expandafter\let\expandafter\saveperiodcentered
	\csname T1\string\textperiodcentered \endcsname

\DeclareTextCommand{\textperiodcentered}{T1}[1]{%
	\ifnum\spacefactor=998
		\Xgem
	\else
		\xgem
	\fi#1}

%-----------------------------------------------------------------
%	PDF INFO AND HYPERREF
%-----------------------------------------------------------------
\usepackage[hypertexnames=false]{hyperref}
\hypersetup{colorlinks, citecolor=black, filecolor=black, linkcolor=black, urlcolor=black}
\usepackage{cleveref}
	\crefname{section}{\S}{\SS}
	\Crefname{section}{\S}{\SS}
	\crefname{listing}{snippet}{}

\newcommand*{\mytitle}{PDS Deliveries}
\newcommand*{\myauthor}{Cristian Estany, Alfredo Hernández, Alejandro Jiménez}
\newcommand*{\myshortauthor}{C. Estany, A. Hernández, A. Jiménez}
\newcommand*{\mydate}{\normalsize \today}

\pdfstringdefDisableCommands{\def\and{and }}

\usepackage{hyperxmp}
\hypersetup{pdfauthor={\myauthor}, pdftitle={\mytitle}}

%-----------------------------------------------------------------
%	DOCUMENT BODY
%-----------------------------------------------------------------
\begin{document}

%\tableofcontents
\thispagestyle{firststyle}
% %-----------------------------------------------------------------
%	DELIVERY 171005
%	!TEX root = ./main.tex
%-----------------------------------------------------------------
\setcounter{section}{0}
\setcounter{page}{1}

\nocite{o:genome}

\section{Delivery} % TODO change \mytitle conveniently
Download \inline{jan2017articles.csv} and \inline{example.bed} files to solve the next questions.

%--------------------
\subsection*{Question 1 (1p)}
Take a look at the last 10 lines of the \inline{jan2017articles.csv} file. Which command are you going to use? Modify the command to show just the last line of the file.

\subsubsection*{Answer}
This can be easily accomplished using \inline{tail} and its \inline{-n} option.

\begin{lstlisting}[language=bash]
# Show last 10 lines
tail jan2017articles.csv
\end{lstlisting}

\begin{lstlisting}[style=output]
04 Jan 2017,Article,Seth Kenlon,By Jove! It's a lightweight alternative to Vim,14,/article/17/1/jove-lightweight-alternative-vim,Text editors,1346
04 Jan 2017,Article,Preston Ward,DronePan: An app that captures panorama views with your aircraft,2,/article/17/1/dronepan,Hardware,533
03 Jan 2017,Article,Shawn Powers,4 hot skills for Linux pros in 2017,10,/article/17/1/yearbook-4-hot-skills-linux-pros-2017,"2016 Open Source Yearbook, Yearbook, Careers",708
03 Jan 2017,Article,Anna Ossowski,What does cross stitch have to do with programming? More than you think,5,/article/17/11/traditional-arts-crafts-code-programming,"JavaScript, Programming",1997
03 Jan 2017,Article,Mihai Raulea,Tapitoo OpenCart: An open source e-commerce mobile app,1,/article/17/1/tapitoo-opencart,Mobile,1127
03 Jan 2017,Article,Sam Knuth,Avoid echo chambers and make open decisions,1,/open-organization/17/1/avoid-echo-chambers-make-open-decisions,The Open Organization,723
02 Jan 2017,Article,Richard Fontana,7 notable legal developments in open source in 2016,3,/article/17/1/yearbook-7-notable-legal-developments-2016,"2016 Open Source Yearbook, Yearbook, Licensing",2359
02 Jan 2017,Article,Scott Nesbitt,3 tips for effectively using wikis for documentation,1,/article/17/1/tips-using-wiki-documentation,"Documentation, Wiki",710
02 Jan 2017,Article,Jen Wike Huger,The Opensource.com preview for January,0,/article/17/1/editorial-preview-january,,358
02 Jan 2017,Poll,Jason Baker,What is your open source New Year's resolution?,1,/poll/17/1/what-your-open-source-new-years-resolution,,186

02 Jan 2017,Poll,Jason Baker,What is your open source New Year's resolution?,1,/poll/17/1/what-your-open-source-new-years-resolution,,186
\end{lstlisting}

\begin{lstlisting}[language=bash]
# Show last line
tail -n1 jan2017articles.csv
\end{lstlisting}

\begin{lstlisting}[style=output]
02 Jan 2017,Poll,Jason Baker,What is your open source New Year's resolution?,1,/poll/17/1/what-your-open-source-new-years-resolution,,186
\end{lstlisting}

%--------------------
\subsection*{Question 2 (1p)}
Extract all lines that belong to January 6th from the file and store them in a new file named \inline{reyes.csv}. Check that the first line of the new file has the expected values.

\subsubsection*{Answer}
This can be done using any tool to search plain-text data, such as \inline{grep} or \inline{awk}. For the sake of practising and learning, we'll use \inline{awk} as much as possible.

\begin{lstlisting}[language=bash]
# First method (could cause some problems)
awk '/06 Jan/ { print $0 }' jan2017articles.csv > reyes1.csv
# Second method (more reliable)
awk '{ if($1 == "06 Jan 2017") print $0 }' FS="," jan2017articles.csv > reyes2.csv
\end{lstlisting}

We'll see however, that in this case both methods are equally valid:
\begin{lstlisting}[language=bash]
# Empty output (both files are the same)
diff reyes1.csv reyes2.csv
\end{lstlisting}

\begin{lstlisting}[style=output, showlines=true]

\end{lstlisting}

Let's check the content of the first line to check if it has the expected values:
\begin{lstlisting}[language=bash]
head -n1 reyes1.csv
\end{lstlisting}

\begin{lstlisting}[style=output]
06 Jan 2017,Article,Jen Wike Huger,"Top 5: Hot programming trends, How Linux got to be Linux, and more",0,/article/17/1/top-5-january-6,Top 5,241
\end{lstlisting}

%--------------------
\subsection*{Question 3 (1p)}
Use the original \inline{csv} to find which entries have \inline{0} at the comment count only for those entries from January 25th.

\subsubsection*{Answer}
The advantages of using \inline{awk} is that we can do more than one operation in a robust and easy to understand way.
\begin{lstlisting}[language=bash]
awk '{ if($1 == "25 Jan 2017" && $5 == 0) print $0 }' FS="," jan2017articles.csv
\end{lstlisting}

\begin{lstlisting}[style=output]
25 Jan 2017,Article,Ben Cotton,24 Pull Requests challenge encourages fruitful contributions,0,/article/17/1/24-Pull-Requests-challenge,Programming,429
25 Jan 2017,Article,Rikki Endsley,Announcing the 2016 Open Source Yearbook: Download now,0,/article/17/1/announcing-2016-open-source-yearbook,2016 Open Source Yearbook,58
\end{lstlisting}

We could've done this by \inline{grep}ping both patterns, but with \inline{awk} we not only search for the patterns, but we specify where to find them, which reduces the possibility of committing any filtering error.

%--------------------
\subsection*{Question 4 (1p)}
Now count the number of entries of Question 3 and compare with the total number of entries.

\subsubsection*{Answer}
We'll simply use \inline{wc} to handle the word-counting and store the number in two different variables.
\begin{lstlisting}[language=bash]
# We count the lines excluding the header
counttot=$(cat jan2017articles.csv | tail -n +2 | wc -l)
# We count the lines of question 3's output
count3=$(awk '{ if($1 == "25 Jan 2017" && $5 == 0) print $0 }' FS="," jan2017articles.csv | wc -l)

# Calculation using a bc (byte code) pipe
echo $counttot - $count3 | bc
\end{lstlisting}

\begin{lstlisting}[style=output]
90
\end{lstlisting}
Let's notice that for comparison we used the difference between both values. We could have just printed both values, but that's probably just a matter of opinion.

%--------------------
% \newpage
\subsection*{Question 5 (1p)}
Now use \inline{example.bed} file. In this file, we are interested in the exon sizes of each entry. They are located in field number 11. Now you have to get the exon sizes of the first 10 entries of the file.

\subsubsection*{Answer}
Being unfamiliar with the format of \inline{.bed} files, we first of all check the data structure~\cite{o:genome}:
\begin{quote}
	11. \textbf{blockSizes} - A comma-separated list of the block sizes. The number of items in this list should correspond to \emph{blockCount}.
\end{quote}
Knowing this, we proceed as instructed.
\begin{lstlisting}[language=bash]
awk '{ print $11 }' example.bed | head
\end{lstlisting}

\begin{lstlisting}[style=output]
541,322,429,
385,143,144,186,125,573,
258,19,143,144,186,125,573,
370,107,97,101,57,77,163,98,80,263,
315,113,97,101,57,77,163,98,101,
370,113,97,101,57,77,163,98,80,257,
370,104,97,101,57,77,163,98,80,263,
370,113,97,101,57,77,163,98,80,263,
293,93,81,72,132,87,72,86,133,189,275,
203,96,81,72,132,87,72,86,133,189,275,
\end{lstlisting}

%--------------------
\subsection*{Question 6 (1p)}
How would you remove the last comma?

\subsubsection*{Answer}
We'll compare different methods using \inline{awk} and \inline{gsub} to substitute patterns/characters in the data.
\begin{lstlisting}[language=bash]
# Using awk and substring (removes last character)
awk '{print substr($11, 1, length($11)-1)}' example.bed | head -n2
# Using awk and gsub (removes last character)
echo
awk '{ gsub(/.$/,"",$11) ; print $11 }' example.bed | head -n2
# Using awk and gsub (removes last comma; probably the best)
echo
awk '{ gsub(/,$/,"",$11) ; print $11 }' example.bed | head -n2
\end{lstlisting}

\begin{lstlisting}[style=output]
541,322,429
385,143,144,186,125,573

541,322,429
385,143,144,186,125,573

541,322,429
385,143,144,186,125,573
\end{lstlisting}

%--------------------
\subsection*{Question 7 (1p)}
How would get the smallest size from each of the records? The result should provide a number for each line of the input.

\subsubsection*{Answer}
This can be done with a classical algorithm to find a minimal value in a list of numbers using \inline{awk}.

\begin{lstlisting}[language=bash]
awk '{ gsub(/,$/,"",$11) ; print $11 }' example.bed | awk '{m=$1; for(i=1;i<=NF;i++) if($i<m) m=$i; print "min of line", NR": ", m}' FS="," | head -n5
\end{lstlisting}

\begin{lstlisting}[style=output]
min of line 1:  322
min of line 2:  125
min of line 3:  19
min of line 4:  57
min of line 5:  57
\end{lstlisting}

%--------------------
\subsection*{Question 8 (1p)}
How would you now sort the records so that the first number shown is the smallest exon size? Again, the answer must provide a sorted list of numbers for each line of the input.

\subsubsection*{Answer}
Although the commands we used in Question 7 are valid for the purpose we wanted to achieve, it involved working exclusively with field 11 and forgetting about the rest of the data file.

We want to work with field 11 within the whole file, so we'll have to rewrite the command a little bit to divide field 11 into an array (\inline{a}), although it does essentially the same. By printing the minimum values in a new field (at the beginning, for instance), we can easily sort the output.
\begin{lstlisting}[language=bash]
awk '{ gsub(/,$/,"",$11); split($11, a, ","); m=$11; for(i=1;i<=length(a);i++) if(a[i]<m) m=a[i]; print m, $0}' OFS="\t" example.bed | sort -n 2>/dev/null | head -n5
\end{lstlisting}

\begin{lstlisting}[style=output]
1	chr3	3628592	3630410	AT3G11530.2	0	-	3628800	3630324	0	5	105,1,52,125,392	1713,1417,1260,471,0,
1	chr4	15669218	15671194	AT4G32470.2	0	-	15669704	15671095	0	5	193,158,48,1,491	1783,1514,852,801,0,
2	chr1	10274047	10275539	AT1G29355.1	0	+	10274047	10275539	0	3	2,697,225	0,346,1267,
2	chr2	14807448	14810164	AT2G35130.1	0	-	14807588	14810164	0	8	2,185,233,250,158,206,125,757	2714,2245,1930,1590,1333,1048,839,0,
2	chr5	1716870	1719541	AT5G05720.1	0	+	1716870	1719541	0	11	2,111,115,33,66,282,66,196,83,220,182	0,117,295,492,690,848,1302,1496,1876,2163,2489,
\end{lstlisting}

Notice that we use \inline{2>/dev/null} in \inline{sed} to redirect the following error:
\begin{lstlisting}[style=output]
sort: write failed: 'standard output': Broken pipe
sort: write error
\end{lstlisting}

This may have something to do with \inline{bash}; \inline{zsh} (the shell I've been using for years) doesn't seem to give this error.

%--------------------
\subsection*{Question 9 (1p)}
Now get the 10 largest exons of \inline{chr1} stored in \inline{example.bed}.

\subsubsection*{Answer}
This is very similar to the previous exercise, but we calculate the maximum of the exons (field 11) and sort using this new field. Notice that we declare \inline{m} as \inline{m=asort(a)} instead of \inline{m=$11} as we did for the minimum calculation for the algorithm to work (not sure why, though; it surely has something to do with the internal structure of the constructed array).

\begin{lstlisting}[language=bash]
awk '/chr1/ { gsub(/,$/,"",$11); split($11, a, ","); m=asort(a); for(i=1;i<=length(a);i++) if(a[i]>m) m=a[i];  print m, $0}' OFS="\t" example.bed | sort -r -n 2>/dev/null | head
\end{lstlisting}

\begin{lstlisting}[style=output]
7713	chr1	26488521	26501281	AT1G70320.1	0	-	26488744	26501281	0	15	33,96,207,7713,318,202,245,251,1108,147,81,226,140,107,326	12727,12423,12128,4271,3877,3522,3194,2850,1584,1355,1163,848,609,397,0,
5616	chr1	28816640	28822256	AT1G76780.1	0	+	28816640	28822256	0	1	5616	0,
5239	chr1	7560564	7565803	AT1G21580.1	0	-	7560564	7565655	0	1	5239	0,
4755	chr1	7773062	7780586	AT1G22060.1	0	-	7773372	7780586	0	9	78,201,123,165,4755,156,102,143,587	7446,7092,6884,6609,1641,1178,904,680,0,
4154	chr1	731703	737332	AT1G03080.1	0	-	731793	737332	0	3	100,4154,1038	5529,1224,0,
4075	chr1	24149542	24154274	AT1G65010.1	0	+	24149542	24154024	0	3	17,196,4075	0,361,657,
3897	chr1	3333594	3337491	AT1G10170.1	0	-	3333924	3337491	0	1	3897	0,
3882	chr1	20879465	20895393	AT1G55860.1	0	-	20879899	20895393	0	19	100,68,612,96,207,3762,3882,321,202,245,269,1108,147,81,226,140,107,188,256	15828,15203,14477,12857,12558,8588,4589,4192,3835,3507,3149,1871,1642,1454,1135, 916,617,349,0,
3875	chr1	4788558	4794654	AT1G13980.1	0	+	4789586	4794397	0	3	96,864,3875	0,902,2221,
3757	chr1	28075073	28078830	AT1G74720.1	0	+	28075172	28078418	0	1	3757	0,
\end{lstlisting}

%--------------------
\subsection*{Question 10 (1p)}
Now modify Question 9 script to receive as a parameter the number of exons to search for.

\subsubsection*{Answer}
To read $N$ as a command line argument, we just need to modify the script following the typical bash syntax:
\renewcommand{\lstlistingname}{Script}
\begin{lstlisting}[caption=Contents of \inline{largest-exons.sh}]
#!/bin/bash

N=$1

awk '/chr1/ { gsub(/,$/,"",$11); split($11, a, ","); m=asort(a); for(i=1;i<=length(a);i++) if(a[i]>m) m=a[i];  print m, $0}' OFS="\t" example.bed | sort -r -n 2>/dev/null | head -n$N
\end{lstlisting}
\renewcommand{\lstlistingname}{Snippet}

%--------------------
% \newpage
\subsection*{Notice}
The \inline{Title} field in \inline{jan2017articles.csv} is not formatted well and may have commas inside it, making really difficult to work with columns/fields. Thankfully, the field separator is just a single comma (\inline{,}), whilst the commas in the titles are accompanied by a space (\inlinev{, }). Knowing this, we just need to systematically perform a \inline{gsub(", ", " ")} substitution before working with this file.

We should have done this as well in questions 3 and 4, but the titles of the matched lines in those questions are properly formatted and don’t induce any error.

%--------------------
\subsection*{Question 11 (1p)}
Get the first 10 records of \inline{jan2017articles.csv} with largest number of comments from the original \inline{csv} file.

\subsubsection*{Answer}
\begin{lstlisting}[language=bash]
awk '{ gsub(", ", " "); print $0}' FS="," jan2017articles.csv | sort -t',' -k5 -n -r 2>/dev/null | head
\end{lstlisting}

\begin{lstlisting}[style=output]
17 Jan 2017,Poll,Opensource.com,What is your favorite Linux distribution?,174,/poll/17/1/what-your-favorite-linux-distribution,"Linux Poll",278
26 Jan 2017,Article,David Both,Using rsync to back up your Linux system,19,/article/17/1/rsync-backup-linux,"Linux Command line",1838
25 Jan 2017,Article,Don Watkins,Solid state drives in Linux: Enabling TRIM for SSDs,19,/article/17/1/solid-state-drives-linux-enabling-trim-ssds,Linux,573
09 Jan 2017,Article,Jeremy Garcia,Troubleshooting tips for the 5 most common Linux issues,18,/article/17/1/yearbook-linux-troubleshooting-tips,"2016 Open Source Yearbook Yearbook The Queue column Linux",794
18 Jan 2017,Article,David Both,10 reasons to use Cinnamon as your Linux desktop environment,14,/article/17/1/cinnamon-desktop-environment,Linux,1296
04 Jan 2017,Article,Seth Kenlon,By Jove! It's a lightweight alternative to Vim,14,/article/17/1/jove-lightweight-alternative-vim,Text editors,1346
04 Jan 2017,Article,Daniel J Walsh,50 ways to avoid getting hacked in 2017,14,/article/17/1/yearbook-50-ways-avoid-getting-hacked,"Yearbook 2016 Open Source Yearbook Security and encryption Containers Docker Linux",2143
30 Jan 2017,Article,David Egts,How to get up and running with sweet Orange Pi,12,/article/17/1/how-to-orange-pi,"Hardware How-tos and tutorials",933
24 Jan 2017,Article,Stefano Maffulli,Brotli: A new compression algorithm for faster Internet,12,/article/17/1/brotli-compression-algorithm,Internet,590
25 Jan 2017,Article,Jen Wike Huger,Happy birthday to Opensource.com: 7 years of open source,11,/article/17/1/happy-birthday-7,Opensource.com community,244
\end{lstlisting}

%--------------------
\subsection*{Question 12 (1p)}
Modify your previous script to receive a number as a parameter $N$ and then show the top $N$ entries with more comments.

As we said in Question 10, we just need to use an argument for $N$:

\renewcommand{\lstlistingname}{Script}
\begin{lstlisting}[caption=Contents of \inline{largest-comments.sh}]
#!/bin/bash

N=$1

awk '{ gsub(", ", " "); print $0}' FS="," jan2017articles.csv | sort -t',' -k5 -n -r 2>/dev/null | head -n$N
\end{lstlisting}
\renewcommand{\lstlistingname}{Snippet}

%--------------------
\subsection*{Question 13 (1p)}
Now we are going to create a new \inline{articles.csv} where we get a different output data layout using \inline{awk} tool:

\begin{lstlisting}[style=output]
INPUT: Post date,Content type,Author,Title,Comment count,Path,Tags,Word count
OUTPUT: Title;Comment count;Word count;Post date
\end{lstlisting}

\subsubsection*{Answer}
\begin{lstlisting}[language=bash]
awk '{ gsub(", ", " "); print $4, $5, $8, $1}' FS="," OFS=";" jan2017articles.csv > articles.csv
head -n5 articles.csv
\end{lstlisting}

\begin{lstlisting}[style=output]
Title;Comment count;Word count;Post date
Book review: Ours to Hack and to Own;0;660;31 Jan 2017
5 new guides for working with OpenStack;2;419;31 Jan 2017
Be the open source supply chain;1;1668;31 Jan 2017
Developing open leaders;1;768;31 Jan 2017
\end{lstlisting}

%--------------------
\subsection*{Question 14 (1p)}
Now create a new \inline{article2.csv} format where we cut the \inline{Title} text to 10 characters and we get only the last level of the \inline{Path}.

\subsubsection*{Answer}
\begin{lstlisting}[language=bash]
awk '{ gsub(", ", " "); a = substr($4, 1, 10) ; split($6, b, "/"); print $1, $2, $3, a, $5, b[length(b)], $7, $8 }' FS="," OFS=";" jan2017articles.csv > article2.csv
head -n5 article2.csv
\end{lstlisting}

\begin{lstlisting}[style=output]
Post date;Content type;Author;Title;Comment count;Path;Tags;Word count
31 Jan 2017;Article;Scott Nesbitt;Book revie;0;review-book-ours-to-hack-and-own;Books;660
31 Jan 2017;Article;Jason Baker;5 new guid;2;openstack-tutorials;"OpenStack How-tos and tutorials";419
31 Jan 2017;Article;John Mark Walker;Be the ope;1;be-open-source-supply-chain;Business;1668
31 Jan 2017;Article;DeLisa Alexander;Developing;1;developing-open-leaders;The Open Organization;768
\end{lstlisting}

% %-----------------------------------------------------------------
%	DELIVERY 171218
%	!TEX root = ./main.tex
%-----------------------------------------------------------------
\setcounter{section}{1}
\setcounter{page}{1}
\section{Delivery} % TODO change \mytitle conveniently
% \subsection*{Objectives}
% \begin{itemize}
% 	\item Learn how to access a MySQL database, look for database and table structure and visualize the content.
% 	\item Learn how to query a MySQL database to extract data following a defined data need. Practice how to translate data query needs into SQL queries.
% \end{itemize}

%--------------------
\subsection*{Question 0}
Can you describe the series of steps to open a database for querying?

\subsubsection*{Answer}
First of all, we need to open a MariaDB\footnote{Arch Linux does not include MySQL in the main repositories, and for the purposes of this course, both database servers are essentially interchangable.} session.

\begin{lstlisting}[language=bash]
# Start service
sudo systemctl start mariadb.service
# Start MariaDB
mysql -u root -p
\end{lstlisting}

\begin{lstlisting}[style=output]
Enter password:
\end{lstlisting}

After entering the password\footnote{A password can be set up using the following command: \inline{mysqladmin -u root password <PASSWORD>}.}, we get the welcome message:

\begin{lstlisting}[style=output]
Welcome to the MariaDB monitor.  Commands end with ; or \g.
Your MariaDB connection id is 3
Server version: 10.1.29-MariaDB MariaDB Server

Copyright (c) 2000, 2017, Oracle, MariaDB Corporation Ab and others.

Type 'help;' or '\h' for help. Type '\c' to clear the current input statement.

MariaDB [(none)]>
\end{lstlisting}

Then we need to check if the the database we are interested in (which we already added to MariaDB in class) is in our list of databases:
\begin{lstlisting}[language=sql]
SHOW databases;
\end{lstlisting}

\begin{lstlisting}[style=output]
+--------------------+
| Database           |
+--------------------+
| experiments        |
| information_schema |
| mysql              |
| performance_schema |
| test               |
+--------------------+
5 rows in set (0.00 sec)
\end{lstlisting}

Then all we do is tell MariaDB to use the \inline{experiments} database:
\begin{lstlisting}[language=sql]
USE experiments;
\end{lstlisting}

\begin{lstlisting}[style=output]
Database changed
\end{lstlisting}

Having changed our database, we can check what tables are contained in it:
\begin{lstlisting}[language=sql]
SHOW Tables;
\end{lstlisting}

\begin{lstlisting}[style=output]
+-----------------------+
| Tables_in_experiments |
+-----------------------+
| Data                  |
| Descriptions          |
| GO_Descr              |
| LocusDescr            |
| LocusLinks            |
| Ontologies            |
| RefSeqs               |
| Sources               |
| Targets               |
| UniDescr              |
| UniSeqs               |
| Unigenes              |
+-----------------------+
12 rows in set (0.01 sec)
\end{lstlisting}

Since we will be using all these tables throughout the exercise, we will summarise the fields contained in each of the tables above:

\begin{table}[H]
	\centering
	\begin{tabular}{l l l l}
		\toprule
		\toprule
		Table                 & \multicolumn{3}{l}{Data fields in the table}    \\
		\midrule
		\texttt{Data}         & \texttt{affyId}, & \texttt{exptId},      & \texttt{level}    \\
		\texttt{Descriptions} & \texttt{gbId},   & \texttt{description}  &                   \\
		\texttt{GO\_Descr}    & \texttt{goAcc},  & \texttt{description}  &                   \\
		\texttt{LocusDescr}   & \texttt{linkId}, & \texttt{description}, & \texttt{species}  \\
		\texttt{LocusLinks}   & \texttt{gbId},   & \texttt{linkId}       &                   \\
		\texttt{Ontologies}   & \texttt{linkId}, & \texttt{goAcc}        &                   \\
		\texttt{RefSeqs}      & \texttt{linkId}, & \texttt{ntRefSeq},    & \texttt{aaRefSeq} \\
		\texttt{Sources}      & \texttt{exptId}, & \texttt{source}       &                   \\
		\texttt{Targets}      & \texttt{gbId},   & \texttt{affyId},      & \texttt{species}  \\
		\texttt{UniDescr}     & \texttt{uId},    & \texttt{description}  &                   \\
		\texttt{UniSeqs}      & \texttt{uId},    & \texttt{gbId}         &                   \\
		\texttt{Unigenes}     & \texttt{linkId}, & \texttt{uId}          &                   \\
		\bottomrule
	\end{tabular}
	\caption{Data fields in the different tables in the \inline{experiments} database}
	\label{tab:tablename}
\end{table}

\newpage
%--------------------
\subsection*{Question 1}
What is the purpose of this query?

\begin{lstlisting}[language=sql]
SELECT * FROM Sources;
\end{lstlisting}

\subsubsection*{Answer}
What this command does is show all the features (columns) from table \inline{Sources}:

\begin{lstlisting}[style=output]
+--------+-------------+
| exptId | source      |
+--------+-------------+
| 1      | Pancreas    |
| 2      | Liver       |
| 4      | Human Liver |
+--------+-------------+
3 rows in set (0.00 sec)
\end{lstlisting}

%--------------------
\subsection*{Question 2}
Get 5 GenBank IDs (\inline{gbId}) and corresponding descriptions.

\subsubsection*{Answer}
We want to select the first 5 records from table \inline{Sources}. We can easily do this using the command \inline{LIMIT}:
\begin{lstlisting}[language=sql]
SELECT * FROM Descriptions
LIMIT 5;
\end{lstlisting}

\begin{lstlisting}[style=output]
+--------+-------------------------------------------+
| gbId   | description                               |
+--------+-------------------------------------------+
| A00142 | granulysin                                |
| A00146 | lypase, gastric                           |
| A03911 | seryne (or cysteine) proteinase inhibitor |
| A06977 | albumin                                   |
| A12027 | S100 calcium binding protein A8           |
+--------+-------------------------------------------+
5 rows in set (0.00 sec)
\end{lstlisting}

%--------------------
\subsection*{Question 3}
What is the purpose of this query?

\begin{lstlisting}[language=sql]
SELECT COUNT(*) FROM LocusLinks;
\end{lstlisting}

\subsubsection*{Answer}
What this command does is count the amount of records (rows) contained in table \inline{LocusLink} and show the resulting number:
\begin{lstlisting}[style=output]
+----------+
| COUNT(*) |
+----------+
|       22 |
+----------+
1 row in set (0.00 sec)
\end{lstlisting}

%--------------------
\subsection*{Question 4}
How many different Affy IDs (\inline{affyId}) are in the expression data?

\subsubsection*{Answer}
To get the amount of different Affy IDs we need to compose \inline{COUNT()} with \inline{DISTINCT()}:

\begin{lstlisting}[language=sql]
SELECT COUNT(DISTINCT(affyId)) FROM Data;
\end{lstlisting}

\begin{lstlisting}[style=output]
+-------------------------+
| COUNT(DISTINCT(affyId)) |
+-------------------------+
|                      23 |
+-------------------------+
1 row in set (0.00 sec)
\end{lstlisting}

%--------------------
\subsection*{Question 5}
What is the expression level of Affy ID \inline{U95-32123_at} in experiment number 1?

\subsubsection*{Answer}
Using the query below, we can see that the expression level is \inline{128}:

\begin{lstlisting}[language=sql]
SELECT * FROM Data
WHERE affyId='U95-32123_at' AND exptId=1;
\end{lstlisting}

\begin{lstlisting}[style=output]
+--------------+--------+-------+
| affyId       | exptId | level |
+--------------+--------+-------+
| U95-32123_at | 1      |   128 |
+--------------+--------+-------+
1 row in set (0.01 sec)
\end{lstlisting}

%--------------------
\subsection*{Question 6}
Find all the gene descriptions, along with their GenBank IDs containing the word “Human”?

\subsubsection*{Answer}
We can do this using \inline{LIKE} when using the \inline{WHERE} clause. Notice that the argument for \inline{LIKE} is case insensitive:

\begin{lstlisting}[language=sql]
SELECT * FROM Descriptions
WHERE description LIKE "%human%";
\end{lstlisting}

\begin{lstlisting}[style=output]
+--------+---------------------------------------------------------+
| gbId   | description                                             |
+--------+---------------------------------------------------------+
| A12345 | HSLFBPS7 Human fructose-1, 6-biphosphatase              |
| A12346 | HSU30872 Human mitosin mRNA                             |
| A12347 | HSU33052 Human lipid-activated protein kinase           |
| A12348 | HSU33053 Human lipid-activated protein kinase           |
| A12349 | Human clone lambda 5 semaphorin mRNA                    |
| A22124 | Human rearranged immunoglobulin lambda light chain mRNA |
| A22127 | Human rearranged immunoglobulin lambda light chain mRNA |
+--------+---------------------------------------------------------+
7 rows in set (0.00 sec)
\end{lstlisting}

%--------------------
\subsection*{Question 7}
What Gene Ontology descriptions (and corresponding accession) contain the phrase “protein kinase”? \emph{The answer should be provided in ascending order of accessions.}


\subsubsection*{Answer}

\begin{lstlisting}[language=sql]
SELECT * FROM GO_Descr
WHERE description LIKE "%protein kinase%"
ORDER BY goAcc ASC;
\end{lstlisting}

\begin{lstlisting}[style=output]
+---------+----------------+
| goAcc   | description    |
+---------+----------------+
| 0001236 | protein kinase |
| 0001237 | protein kinase |
| 1112222 | protein kinase |
| 4442222 | protein kinase |
+---------+----------------+
4 rows in set (0.00 sec)
\end{lstlisting}

%--------------------
\subsection*{Question 8}
Which AffyId of table \inline{Data} correspond to sequences in \inline{Targets} table with the phrase “kinase” in their description?

\subsubsection*{Answer}

\begin{lstlisting}[language=sql]
SELECT Data.affyId, Descriptions.description
FROM Data, Targets, Descriptions
WHERE Data.affyId=Targets.affyId AND
      Targets.gbId=Descriptions.gbId AND
      Descriptions.description LIKE "%kinase%";
\end{lstlisting}

\begin{lstlisting}[style=output]
Empty set (0.01 sec)
\end{lstlisting}

Now we use the command
\begin{lstlisting}[language=sql]
LOAD DATA INFILE 'file.tsv' INTO TABLE Descriptions;
\end{lstlisting}
to add two new entries in the \inline{Descriptions} table with \inline{gbId="M18228", "L02870"} and the string “kinase” as the description for both, where \inline{file.tsv} has the folliwing contents:
\begin{lstlisting}[language={}, showtabs=true]
M18228	kinase
L02870	kinase
\end{lstlisting}

Now we repeat the query again:
\begin{lstlisting}[language=sql]
SELECT Data.affyId, Descriptions.description
FROM Data, Targets, Descriptions
WHERE Data.affyId=Targets.affyId AND
      Targets.gbId=Descriptions.gbId AND
      Descriptions.description LIKE "%kinase%";
\end{lstlisting}

\begin{lstlisting}[style=output]
Empty set (0.00 sec)
\end{lstlisting}

The idea is that by adding the new records into the \inline{Description} table, we would obtain some results. The problem is that the data is either incomplete or not well-formatted (something we would have to report to the database architect).

For instance, for the first line (\inline{M18228	kinase}) we have done the following test:
\begin{lstlisting}[language=sql]
SELECT * FROM Descriptions WHERE gbId LIKE "M18228";
SELECT * FROM Targets WHERE gbId LIKE "M18228";
SELECT * FROM Data WHERE affyId LIKE "U95_32123_at";
\end{lstlisting}

\begin{lstlisting}[style=output]
MariaDB [experiments]> SELECT * FROM Descriptions WHERE gbId LIKE "M18228";
+--------+-------------+
| gbId   | description |
+--------+-------------+
| M18228 | kinase      |
+--------+-------------+
1 row in set (0.00 sec)

MariaDB [experiments]> SELECT * FROM Targets WHERE gbId LIKE "M18228";
+--------+---------+---------+
| gbId   | affyId  | species |
+--------+---------+---------+
| M18228 | 1235_at | Mm      |
+--------+---------+---------+
1 row in set (0.00 sec)

MariaDB [experiments]> SELECT * FROM Data WHERE affyId LIKE "%1235%";
Empty set (0.01 sec)
\end{lstlisting}
Notice how there is no \inline{1235_at} \inline{affyId} in the table \inline{Data}.

For the second line (\inline{L02870	kinase}), the problem is how the \inline{affyId} is formatted:

\begin{lstlisting}[language=sql]
SELECT * FROM Targets WHERE gbId LIKE "L02870";
SELECT * FROM Data WHERE affyId LIKE "U95_32123_at";
\end{lstlisting}

\begin{lstlisting}[style=output]
MariaDB [experiments]> SELECT * FROM Targets WHERE gbId LIKE "L02870";
+--------+--------------+---------+
| gbId   | affyId       | species |
+--------+--------------+---------+
| L02870 | U95_32123_at | Mm      |
+--------+--------------+---------+
1 row in set (0.00 sec)

MariaDB [experiments]> SELECT * FROM Data WHERE affyId LIKE "U95_32123_at";
+--------------+--------+-------+
| affyId       | exptId | level |
+--------------+--------+-------+
| U95-32123_at | 1      |   128 |
| U95-32123_at | 2      |   128 |
+--------------+--------+-------+
2 rows in set (0.00 sec)

\end{lstlisting}
Notice how in \inline{Targets}, the \inline{affyId} is formatted as \inline{U95_32123_at}, whilst in \inline{Data} they are expressed as \inline{U95-32123_at}. This causes the query \inline{WHERE Data.affyId=Targets.affyId} to return no output.

\bigskip
We could use the query below instead, but it is far from being a good practice, and it should be avoided:
\begin{lstlisting}[language=sql]
SELECT Data.affyId, Descriptions.description
FROM Data, Targets, Descriptions
WHERE Data.affyId LIKE Targets.affyId AND
      Targets.gbId=Descriptions.gbId AND
      Descriptions.description LIKE "%kinase%";
\end{lstlisting}

\begin{lstlisting}[style=output]
+--------------+-------------+
| affyId       | description |
+--------------+-------------+
| U95-32123_at | kinase      |
| U95-32123_at | kinase      |
+--------------+-------------+
2 rows in set (0.00 sec)
\end{lstlisting}

%--------------------
\subsection*{Question 9}
Get two \inline{affyId}, \inline{uId} and \inline{description} in \inline{LocusDescr} in reverse alphabetical order of descriptions.

\subsubsection*{Answer}
This question can be a little bit tricky, as there is no indication of the specific origin of the \inline{affyId} or \inline{uId}. We will show three different implementations that give us significantly different results:

\begin{lstlisting}[language=sql]
SELECT Data.affyId, UniDescr.uId, LocusDescr.description
FROM Data, Targets, UniDescr, LocusDescr
WHERE Data.affyId=Targets.affyId AND
      Targets.species=LocusDescr.species AND
      LocusDescr.description=UniDescr.description
ORDER BY LocusDescr.description DESC
LIMIT 2;
\end{lstlisting}

\begin{lstlisting}[style=output]
+---------+--------+-------------+
| affyId  | uId    | description |
+---------+--------+-------------+
| 5324_at | Hs1691 | Glucan      |
| 5323_at | Hs1691 | Glucan      |
+---------+--------+-------------+
2 rows in set (0.00 sec)
\end{lstlisting}

\begin{lstlisting}[language=sql]
SELECT Targets.affyId, UniDescr.uId, LocusDescr.description
FROM Targets, LocusDescr, UniDescr
WHERE LocusDescr.species=Targets.species AND
      LocusDescr.description=UniDescr.description
ORDER BY LocusDescr.description DESC
LIMIT 2;
\end{lstlisting}

\begin{lstlisting}[style=output]
+---------+--------+-------------+
| affyId  | uId    | description |
+---------+--------+-------------+
| 3156_at | Hs1691 | Glucan      |
| 5323_at | Hs1691 | Glucan      |
+---------+--------+-------------+
2 rows in set (0.00 sec)
\end{lstlisting}

\begin{lstlisting}[language=sql]
SELECT Targets.affyId, UniSeqs.uId, LocusDescr.description
FROM Targets, UniSeqs, LocusDescr, LocusLinks
WHERE LocusLinks.linkId=LocusDescr.linkId AND
      LocusLinks.gbId=UniSeqs.gbId AND
      Targets.gbId=LocusLinks.gbId
ORDER BY LocusDescr.description DESC
LIMIT 2;
\end{lstlisting}

\begin{lstlisting}[style=output]
+--------------+--------+-------------+
| affyId       | uId    | description |
+--------------+--------+-------------+
| U95_40474_at | Hs1691 | Glucan      |
| U95_32123_at | Hs1640 | Collagen    |
+--------------+--------+-------------+
2 rows in set (0.00 sec)
\end{lstlisting}

The thing to learn from this exercise is that one must be familiar with the database and be aware of the nature of the query.

%--------------------
\subsection*{Question 10}
How would you find the average expression level of each experiment in \inline{Data}?

\subsubsection*{Answer}

\begin{lstlisting}[language=sql]
SELECT exptId, AVG(level) FROM Data
GROUP BY exptId;
\end{lstlisting}

\begin{lstlisting}[style=output]
+--------+------------+
| exptId | AVG(level) |
+--------+------------+
| 1      |   125.3333 |
| 2      |    95.3333 |
| 3      |   126.3333 |
| 4      |    83.5000 |
| 5      |    92.7500 |
| 6      |    18.3333 |
| 7      |    20.0000 |
| 8      |    40.0000 |
| 9      |    20.0000 |
+--------+------------+
9 rows in set (0.00 sec)
\end{lstlisting}

%--------------------
\subsection*{Question 11}
What is the average expression level of each array probe (\inline{affyId}) across all experiments?

\subsubsection*{Answer}

\begin{lstlisting}[language=sql]
SELECT affyId, AVG(level) FROM Data
GROUP BY affyId;
\end{lstlisting}

\begin{lstlisting}[style=output]
+---------------------------+------------+
| affyId                    | AVG(level) |
+---------------------------+------------+
| 31315_at                  |   250.0000 |
| 31324_at                  |    91.0000 |
| 31325_at                  |    89.0000 |
| 31356_at                  |    91.0000 |
| 31362_at                  |   260.0000 |
| 31510_s_at                |   257.0000 |
| 5321_at                   |    90.0000 |
| 5322_at                   |    90.0000 |
| 5323_at                   |    90.0000 |
| 5324_at                   |    73.5000 |
| 5325_at                   |    90.0000 |
| AFFX-BioB-3_at            |    97.0000 |
| AFFX-BioB-5_at            |    20.0000 |
| AFFX-BioB-M_at            |    62.8000 |
| AFFX-HSAC07/X00351_M_at   |    86.0000 |
| AFFX-HUMBAPDH/M33197_3_st |   277.0000 |
| AFFX-HUMTFFR/M11507_at    |    90.0000 |
| AFFX-M27830_3_at          |   271.0000 |
| AFFX-MurIL10_at           |     6.6667 |
| AFFX-MurIL2_at            |    20.0000 |
| AFFX-MurIL4_at            |    49.0000 |
| U95-32123_at              |   128.0000 |
| U98-40474_at              |    57.0000 |
+---------------------------+------------+
23 rows in set (0.00 sec)
\end{lstlisting}


%--------------------
\subsection*{Question 12}
What is the purpose of the following query?

\begin{lstlisting}[language=sql]
SELECT Data.affyId, Data.level, Data.exptId, DataCopy.affyId, DataCopy.level, DataCopy.exptId
FROM Data, Data DataCopy
WHERE Data.level > 10 * DataCopy.level AND
      Data.affyId=DataCopy.affyId AND
      Data.affyId LIKE "AFFX%"
LIMIT 10;
\end{lstlisting}

\subsubsection*{Answer}

\begin{lstlisting}[style=output]
+----------------+-------+--------+----------------+-------+--------+
| affyId         | level | exptId | affyId         | level | exptId |
+----------------+-------+--------+----------------+-------+--------+
| AFFX-BioB-M_at |   214 | 5      | AFFX-BioB-M_at |    20 | 3      |
| AFFX-BioB-M_at |   214 | 5      | AFFX-BioB-M_at |    20 | 7      |
| AFFX-BioB-M_at |   214 | 5      | AFFX-BioB-M_at |    20 | 9      |
+----------------+-------+--------+----------------+-------+--------+
3 rows in set (0.00 sec)
\end{lstlisting}

What this query is doing is to make a copy of the \inline{Data} table in order to compare the \inline{affyId}s that start with \inline{AFFX} to show the pairs that fulfil the condition of having an expression \texttt{level} of one being 10 times larger than the other one. The query lets us not only compare the expression \texttt{level}, but the \inline{exptId} value as well.

% More specifically, the query selects the columns \inline{affyId}, \texttt{level} and \inline{exptId} from \inline{Data} and also the same ones from \inline{DataCopy}, which is an exact copy of table \inline{Data}. This copy has been done in the \inline{FROM} clause.

% In the \inline{WHERE} clause, 3 conditions are introduced:
% \begin{enumerate}[(i)]
% 	\item Select those rows in \texttt{level} from\inline{Data} table which are more than 10 times higher than the rows in \texttt{level} from the copy of\inline{Data} table.
% 	\item Select those rows in which \inline{affyId} from tables\inline{Data} and it's copy coincide.
% 	\item Take those entries in \inline{affyId} from\inline{Data} that start with \inline{AFFX}.
% \end{enumerate}

The \inline{LIMIT 10} clause is there to show only the 10 first results, although it is unnecessary for this specific query, as there are only 3 results.

%--------------------
\subsection*{Question 13}
Write a query to provide three different descriptions for all \inline{gbId} in table \inline{Targets}.

\subsubsection*{Answer}

The tricky part with creating this query is that we have 4 different descriptions (in tables \inline{Descriptions}, \inline{LocusDescr}, \inline{UniDescr}, and \inline{GO_Descr}), so depending on which ones we choose, we will have different results.

Apart from that, one has to be careful on using the right relational path to connect all the tables properly. We will show two queries: (a) one to show \inline{Descriptions.description}, \inline{LocusDescr.description}, and \inline{UniDescr.description}; and (b) another one to show \inline{Descriptions.description}, \inline{GO_Descr.description}, \inline{LocusDescr.description}.

For space reasons, we will only show a sample of the output.

\begin{lstlisting}[language=sql]
SELECT Targets.gbId, Descriptions.description AS description, LocusDescr.description AS LocusDescr, UniDescr.description AS UniDescr
FROM Targets, Descriptions, LocusDescr, UniDescr, UniSeqs
WHERE Targets.gbId=Descriptions.gbId AND
      Targets.species=LocusDescr.species AND
      UniSeqs.uId=UniDescr.uId AND
      Targets.gbId=UniSeqs.gbId
ORDER BY gbId;
\end{lstlisting}

\begin{lstlisting}[style=output]
+--------+-------------+------------+----------------+
| gbId   | description | LocusDescr | UniDescr       |
+--------+-------------+------------+----------------+
| S75295 | Glucan      | Collagen   | Glucan         |
| S75295 | Glucan      | serine     | Glucan         |
| S75295 | Glucan      | Glucan     | Glucan         |
| S75295 | Glucan      | albumin    | Glucan         |
| S75295 | Glucan      | granulysi  | Glucan         |
| ...    | ...         | ...        | ....           |
+--------+-------------+------------+----------------+
42 rows in set (0.00 sec)
\end{lstlisting}

\begin{lstlisting}[language=sql]
SELECT Targets.gbId, Descriptions.description AS description, GO_Descr.description AS GO_Descr, LocusDescr.description AS LocusDescr
FROM Targets, Descriptions, GO_Descr, LocusDescr, Ontologies, LocusLinks
WHERE Descriptions.gbId=Targets.gbId AND
      GO_Descr.goAcc=Ontologies.goAcc AND
      LocusDescr.linkId=Ontologies.linkId AND
      Targets.gbId=LocusLinks.gbId
ORDER BY gbId;
\end{lstlisting}

\begin{lstlisting}[style=output]
+--------+-----------------+--------------+------------+
| gbId   | description     | GoDescr      | LocusDescr |
+--------+-----------------+--------------+------------+
| A00142 | granulysin      | Glucan Enz   | Glucan     |
| A00142 | granulysin      | Glucan Enz   | Glucan     |
| A00142 | granulysin      | Serine Prot. | Collagen   |
| A00142 | granulysin      | Serine Prot. | Collagen   |
| A00146 | lypase, gastric | Glucan Enz   | Glucan     |
| ...    | ...             | ...          | ....       |
+--------+-----------------+--------------+------------+
40 rows in set (0.00 sec)
\end{lstlisting}

% \begin{lstlisting}[language=sql]
% SELECT Targets.gbId, Descriptions.description AS description, GO_Descr.description AS GO_Descr, LocusDescr.description AS LocusDescr
% FROM Targets, Descriptions, Ontologies, GO_Descr, LocusLinks, LocusDescr
% WHERE Descriptions.gbId=Targets.gbId AND
%       Targets.gbId=LocusLinks.gbId AND
%       LocusLinks.linkId=LocusDescr.linkId AND
%       LocusLinks.linkId=Ontologies.linkId AND
%       Ontologies.goAcc=GO_Descr.goAcc
% ORDER BY gbId;
% \end{lstlisting}

% \begin{lstlisting}[style=output]
% +--------+-------------+--------------+------------+
% | gbId   | description | GO_Descr     | LocusDescr |
% +--------+-------------+--------------+------------+
% | L02870 | kinase      | Serine Prot. | Collagen   |
% | S75295 | Glucan      | Glucan Enz   | Glucan     |
% +--------+-------------+--------------+------------+
% 2 rows in set (0.00 sec)
% \end{lstlisting}

%--------------------
\subsection*{Question 14}
Write a query to provide all gene ontology (\inline{GO_descr}) descriptions related with all  species in table \inline{Targets} sorted alphabetically and providing the first five results. Export the query to a tab-separated-file with the command:

\begin{lstlisting}[language=sql]
SELECT * FROM TABLE INTO OUTFILE 'data.out';
\end{lstlisting}

\subsubsection*{Answer}
We just need to do the usual query and add the \inline{INTO OUTFILE} expression at the end of it, as it can be seen below.
\begin{lstlisting}[language=sql]
SELECT Targets.species, GO_Descr.description
FROM Targets, GO_Descr, Ontologies, LocusDescr
WHERE Targets.species=LocusDescr.species AND
      LocusDescr.linkId=Ontologies.linkId AND
      Ontologies.goAcc=GO_Descr.goAcc
ORDER BY species
INTO OUTFILE 'data.out';
\end{lstlisting}

\begin{lstlisting}[style=output]
+---------+--------------+
| species | description  |
+---------+--------------+
| Hs      | Serine Prot. |
| Hs      | Serine Prot. |
| Hs      | Serine Prot. |
| Hs      | Serine Prot. |
| Hs      | Serine Prot. |
+---------+--------------+
5 rows in set (0.00 sec)
\end{lstlisting}


%-----------------------------------------------------------------
%	DELIVERY 180211
%	!TEX root = ./main-group.tex
%-----------------------------------------------------------------
\setcounter{section}{2}
\setcounter{page}{1}
\section{Delivery: Hadoop} % TODO change \mytitle conveniently
% \subsection*{Objectives}
% \begin{itemize}
% 	\item Learn how to access a MySQL database, look for database and table structure and visualize the content.
% 	\item Learn how to query a MySQL database to extract data following a defined data need. Practice how to translate data query needs into SQL queries.
% \end{itemize}

%--------------------
\subsection*{Question 0}
Can you describe the series of steps to open a database for querying?

\subsubsection*{Answer}
First of all, we need to open a MariaDB\footnote{Arch Linux does not include MySQL in the main repositories, and for the purposes of this course, both database servers are essentially interchangable.} session.

\begin{lstlisting}[language=bash]
# Start service
sudo systemctl start mariadb.service
# Start MariaDB
mysql -u root -p
\end{lstlisting}

\begin{lstlisting}[style=output]
Enter password:
\end{lstlisting}

After entering the password\footnote{A password can be set up using the following command: \inline{mysqladmin -u root password <PASSWORD>}.}, we get the welcome message:

\begin{lstlisting}[style=output]
Welcome to the MariaDB monitor.  Commands end with ; or \g.
Your MariaDB connection id is 3
Server version: 10.1.29-MariaDB MariaDB Server

Copyright (c) 2000, 2017, Oracle, MariaDB Corporation Ab and others.

Type 'help;' or '\h' for help. Type '\c' to clear the current input statement.

MariaDB [(none)]>
\end{lstlisting}

Then we need to check if the the database we are interested in (which we already added to MariaDB in class) is in our list of databases:
\begin{lstlisting}[language=sql]
SHOW databases;
\end{lstlisting}

\begin{lstlisting}[style=output]
+--------------------+
| Database           |
+--------------------+
| experiments        |
| information_schema |
| mysql              |
| performance_schema |
| test               |
+--------------------+
5 rows in set (0.00 sec)
\end{lstlisting}

Then all we do is tell MariaDB to use the \inline{experiments} database:
\begin{lstlisting}[language=sql]
USE experiments;
\end{lstlisting}

\begin{lstlisting}[style=output]
Database changed
\end{lstlisting}

Having changed our database, we can check what tables are contained in it:
\begin{lstlisting}[language=sql]
SHOW Tables;
\end{lstlisting}

\begin{lstlisting}[style=output]
+-----------------------+
| Tables_in_experiments |
+-----------------------+
| Data                  |
| Descriptions          |
| GO_Descr              |
| LocusDescr            |
| LocusLinks            |
| Ontologies            |
| RefSeqs               |
| Sources               |
| Targets               |
| UniDescr              |
| UniSeqs               |
| Unigenes              |
+-----------------------+
12 rows in set (0.01 sec)
\end{lstlisting}

Since we will be using all these tables throughout the exercise, we will summarise the fields contained in each of the tables above:

\begin{table}[H]
	\centering
	\begin{tabular}{l l l l}
		\toprule
		\toprule
		Table                 & \multicolumn{3}{l}{Data fields in the table}    \\
		\midrule
		\texttt{Data}         & \texttt{affyId}, & \texttt{exptId},      & \texttt{level}    \\
		\texttt{Descriptions} & \texttt{gbId},   & \texttt{description}  &                   \\
		\texttt{GO\_Descr}    & \texttt{goAcc},  & \texttt{description}  &                   \\
		\texttt{LocusDescr}   & \texttt{linkId}, & \texttt{description}, & \texttt{species}  \\
		\texttt{LocusLinks}   & \texttt{gbId},   & \texttt{linkId}       &                   \\
		\texttt{Ontologies}   & \texttt{linkId}, & \texttt{goAcc}        &                   \\
		\texttt{RefSeqs}      & \texttt{linkId}, & \texttt{ntRefSeq},    & \texttt{aaRefSeq} \\
		\texttt{Sources}      & \texttt{exptId}, & \texttt{source}       &                   \\
		\texttt{Targets}      & \texttt{gbId},   & \texttt{affyId},      & \texttt{species}  \\
		\texttt{UniDescr}     & \texttt{uId},    & \texttt{description}  &                   \\
		\texttt{UniSeqs}      & \texttt{uId},    & \texttt{gbId}         &                   \\
		\texttt{Unigenes}     & \texttt{linkId}, & \texttt{uId}          &                   \\
		\bottomrule
	\end{tabular}
	\caption{Data fields in the different tables in the \inline{experiments} database}
	\label{tab:tablename}
\end{table}

\newpage
%--------------------
\subsection*{Question 1}
What is the purpose of this query?

\begin{lstlisting}[language=sql]
SELECT * FROM Sources;
\end{lstlisting}

\subsubsection*{Answer}
What this command does is show all the features (columns) from table \inline{Sources}:

\begin{lstlisting}[style=output]
+--------+-------------+
| exptId | source      |
+--------+-------------+
| 1      | Pancreas    |
| 2      | Liver       |
| 4      | Human Liver |
+--------+-------------+
3 rows in set (0.00 sec)
\end{lstlisting}

%--------------------
\subsection*{Question 2}
Get 5 GenBank IDs (\inline{gbId}) and corresponding descriptions.

\subsubsection*{Answer}
We want to select the first 5 records from table \inline{Sources}. We can easily do this using the command \inline{LIMIT}:
\begin{lstlisting}[language=sql]
SELECT * FROM Descriptions
LIMIT 5;
\end{lstlisting}

\begin{lstlisting}[style=output]
+--------+-------------------------------------------+
| gbId   | description                               |
+--------+-------------------------------------------+
| A00142 | granulysin                                |
| A00146 | lypase, gastric                           |
| A03911 | seryne (or cysteine) proteinase inhibitor |
| A06977 | albumin                                   |
| A12027 | S100 calcium binding protein A8           |
+--------+-------------------------------------------+
5 rows in set (0.00 sec)
\end{lstlisting}

%--------------------
\subsection*{Question 3}
What is the purpose of this query?

\begin{lstlisting}[language=sql]
SELECT COUNT(*) FROM LocusLinks;
\end{lstlisting}

\subsubsection*{Answer}
What this command does is count the amount of records (rows) contained in table \inline{LocusLink} and show the resulting number:
\begin{lstlisting}[style=output]
+----------+
| COUNT(*) |
+----------+
|       22 |
+----------+
1 row in set (0.00 sec)
\end{lstlisting}

%--------------------
\subsection*{Question 4}
How many different Affy IDs (\inline{affyId}) are in the expression data?

\subsubsection*{Answer}
To get the amount of different Affy IDs we need to compose \inline{COUNT()} with \inline{DISTINCT()}:

\begin{lstlisting}[language=sql]
SELECT COUNT(DISTINCT(affyId)) FROM Data;
\end{lstlisting}

\begin{lstlisting}[style=output]
+-------------------------+
| COUNT(DISTINCT(affyId)) |
+-------------------------+
|                      23 |
+-------------------------+
1 row in set (0.00 sec)
\end{lstlisting}

%--------------------
\subsection*{Question 5}
What is the expression level of Affy ID \inline{U95-32123_at} in experiment number 1?

\subsubsection*{Answer}
Using the query below, we can see that the expression level is \inline{128}:

\begin{lstlisting}[language=sql]
SELECT * FROM Data
WHERE affyId='U95-32123_at' AND exptId=1;
\end{lstlisting}

\begin{lstlisting}[style=output]
+--------------+--------+-------+
| affyId       | exptId | level |
+--------------+--------+-------+
| U95-32123_at | 1      |   128 |
+--------------+--------+-------+
1 row in set (0.01 sec)
\end{lstlisting}

%--------------------
\subsection*{Question 6}
Find all the gene descriptions, along with their GenBank IDs containing the word “Human”?

\subsubsection*{Answer}
We can do this using \inline{LIKE} when using the \inline{WHERE} clause. Notice that the argument for \inline{LIKE} is case insensitive:

\begin{lstlisting}[language=sql]
SELECT * FROM Descriptions
WHERE description LIKE "%human%";
\end{lstlisting}

\begin{lstlisting}[style=output]
+--------+---------------------------------------------------------+
| gbId   | description                                             |
+--------+---------------------------------------------------------+
| A12345 | HSLFBPS7 Human fructose-1, 6-biphosphatase              |
| A12346 | HSU30872 Human mitosin mRNA                             |
| A12347 | HSU33052 Human lipid-activated protein kinase           |
| A12348 | HSU33053 Human lipid-activated protein kinase           |
| A12349 | Human clone lambda 5 semaphorin mRNA                    |
| A22124 | Human rearranged immunoglobulin lambda light chain mRNA |
| A22127 | Human rearranged immunoglobulin lambda light chain mRNA |
+--------+---------------------------------------------------------+
7 rows in set (0.00 sec)
\end{lstlisting}

%--------------------
\subsection*{Question 7}
What Gene Ontology descriptions (and corresponding accession) contain the phrase “protein kinase”? \emph{The answer should be provided in ascending order of accessions.}


\subsubsection*{Answer}

\begin{lstlisting}[language=sql]
SELECT * FROM GO_Descr
WHERE description LIKE "%protein kinase%"
ORDER BY goAcc ASC;
\end{lstlisting}

\begin{lstlisting}[style=output]
+---------+----------------+
| goAcc   | description    |
+---------+----------------+
| 0001236 | protein kinase |
| 0001237 | protein kinase |
| 1112222 | protein kinase |
| 4442222 | protein kinase |
+---------+----------------+
4 rows in set (0.00 sec)
\end{lstlisting}

%--------------------
\subsection*{Question 8}
Which AffyId of table \inline{Data} correspond to sequences in \inline{Targets} table with the phrase “kinase” in their description?

\subsubsection*{Answer}

\begin{lstlisting}[language=sql]
SELECT Data.affyId, Descriptions.description
FROM Data, Targets, Descriptions
WHERE Data.affyId=Targets.affyId AND
      Targets.gbId=Descriptions.gbId AND
      Descriptions.description LIKE "%kinase%";
\end{lstlisting}

\begin{lstlisting}[style=output]
Empty set (0.01 sec)
\end{lstlisting}

Now we use the command
\begin{lstlisting}[language=sql]
LOAD DATA INFILE 'file.tsv' INTO TABLE Descriptions;
\end{lstlisting}
to add two new entries in the \inline{Descriptions} table with \inline{gbId="M18228", "L02870"} and the string “kinase” as the description for both, where \inline{file.tsv} has the folliwing contents:
\begin{lstlisting}[language={}, showtabs=true]
M18228	kinase
L02870	kinase
\end{lstlisting}

Now we repeat the query again:
\begin{lstlisting}[language=sql]
SELECT Data.affyId, Descriptions.description
FROM Data, Targets, Descriptions
WHERE Data.affyId=Targets.affyId AND
      Targets.gbId=Descriptions.gbId AND
      Descriptions.description LIKE "%kinase%";
\end{lstlisting}

\begin{lstlisting}[style=output]
Empty set (0.00 sec)
\end{lstlisting}

The idea is that by adding the new records into the \inline{Description} table, we would obtain some results. The problem is that the data is either incomplete or not well-formatted (something we would have to report to the database architect).

For instance, for the first line (\inline{M18228	kinase}) we have done the following test:
\begin{lstlisting}[language=sql]
SELECT * FROM Descriptions WHERE gbId LIKE "M18228";
SELECT * FROM Targets WHERE gbId LIKE "M18228";
SELECT * FROM Data WHERE affyId LIKE "U95_32123_at";
\end{lstlisting}

\begin{lstlisting}[style=output]
MariaDB [experiments]> SELECT * FROM Descriptions WHERE gbId LIKE "M18228";
+--------+-------------+
| gbId   | description |
+--------+-------------+
| M18228 | kinase      |
+--------+-------------+
1 row in set (0.00 sec)

MariaDB [experiments]> SELECT * FROM Targets WHERE gbId LIKE "M18228";
+--------+---------+---------+
| gbId   | affyId  | species |
+--------+---------+---------+
| M18228 | 1235_at | Mm      |
+--------+---------+---------+
1 row in set (0.00 sec)

MariaDB [experiments]> SELECT * FROM Data WHERE affyId LIKE "%1235%";
Empty set (0.01 sec)
\end{lstlisting}
Notice how there is no \inline{1235_at} \inline{affyId} in the table \inline{Data}.

For the second line (\inline{L02870	kinase}), the problem is how the \inline{affyId} is formatted:

\begin{lstlisting}[language=sql]
SELECT * FROM Targets WHERE gbId LIKE "L02870";
SELECT * FROM Data WHERE affyId LIKE "U95_32123_at";
\end{lstlisting}

\begin{lstlisting}[style=output]
MariaDB [experiments]> SELECT * FROM Targets WHERE gbId LIKE "L02870";
+--------+--------------+---------+
| gbId   | affyId       | species |
+--------+--------------+---------+
| L02870 | U95_32123_at | Mm      |
+--------+--------------+---------+
1 row in set (0.00 sec)

MariaDB [experiments]> SELECT * FROM Data WHERE affyId LIKE "U95_32123_at";
+--------------+--------+-------+
| affyId       | exptId | level |
+--------------+--------+-------+
| U95-32123_at | 1      |   128 |
| U95-32123_at | 2      |   128 |
+--------------+--------+-------+
2 rows in set (0.00 sec)

\end{lstlisting}
Notice how in \inline{Targets}, the \inline{affyId} is formatted as \inline{U95_32123_at}, whilst in \inline{Data} they are expressed as \inline{U95-32123_at}. This causes the query \inline{WHERE Data.affyId=Targets.affyId} to return no output.

\bigskip
We could use the query below instead, but it is far from being a good practice, and it should be avoided:
\begin{lstlisting}[language=sql]
SELECT Data.affyId, Descriptions.description
FROM Data, Targets, Descriptions
WHERE Data.affyId LIKE Targets.affyId AND
      Targets.gbId=Descriptions.gbId AND
      Descriptions.description LIKE "%kinase%";
\end{lstlisting}

\begin{lstlisting}[style=output]
+--------------+-------------+
| affyId       | description |
+--------------+-------------+
| U95-32123_at | kinase      |
| U95-32123_at | kinase      |
+--------------+-------------+
2 rows in set (0.00 sec)
\end{lstlisting}

%--------------------
\subsection*{Question 9}
Get two \inline{affyId}, \inline{uId} and \inline{description} in \inline{LocusDescr} in reverse alphabetical order of descriptions.

\subsubsection*{Answer}
This question can be a little bit tricky, as there is no indication of the specific origin of the \inline{affyId} or \inline{uId}. We will show three different implementations that give us significantly different results:

\begin{lstlisting}[language=sql]
SELECT Data.affyId, UniDescr.uId, LocusDescr.description
FROM Data, Targets, UniDescr, LocusDescr
WHERE Data.affyId=Targets.affyId AND
      Targets.species=LocusDescr.species AND
      LocusDescr.description=UniDescr.description
ORDER BY LocusDescr.description DESC
LIMIT 2;
\end{lstlisting}

\begin{lstlisting}[style=output]
+---------+--------+-------------+
| affyId  | uId    | description |
+---------+--------+-------------+
| 5324_at | Hs1691 | Glucan      |
| 5323_at | Hs1691 | Glucan      |
+---------+--------+-------------+
2 rows in set (0.00 sec)
\end{lstlisting}

\begin{lstlisting}[language=sql]
SELECT Targets.affyId, UniDescr.uId, LocusDescr.description
FROM Targets, LocusDescr, UniDescr
WHERE LocusDescr.species=Targets.species AND
      LocusDescr.description=UniDescr.description
ORDER BY LocusDescr.description DESC
LIMIT 2;
\end{lstlisting}

\begin{lstlisting}[style=output]
+---------+--------+-------------+
| affyId  | uId    | description |
+---------+--------+-------------+
| 3156_at | Hs1691 | Glucan      |
| 5323_at | Hs1691 | Glucan      |
+---------+--------+-------------+
2 rows in set (0.00 sec)
\end{lstlisting}

\begin{lstlisting}[language=sql]
SELECT Targets.affyId, UniSeqs.uId, LocusDescr.description
FROM Targets, UniSeqs, LocusDescr, LocusLinks
WHERE LocusLinks.linkId=LocusDescr.linkId AND
      LocusLinks.gbId=UniSeqs.gbId AND
      Targets.gbId=LocusLinks.gbId
ORDER BY LocusDescr.description DESC
LIMIT 2;
\end{lstlisting}

\begin{lstlisting}[style=output]
+--------------+--------+-------------+
| affyId       | uId    | description |
+--------------+--------+-------------+
| U95_40474_at | Hs1691 | Glucan      |
| U95_32123_at | Hs1640 | Collagen    |
+--------------+--------+-------------+
2 rows in set (0.00 sec)
\end{lstlisting}

The thing to learn from this exercise is that one must be familiar with the database and be aware of the nature of the query.

%--------------------
\subsection*{Question 10}
How would you find the average expression level of each experiment in \inline{Data}?

\subsubsection*{Answer}

\begin{lstlisting}[language=sql]
SELECT exptId, AVG(level) FROM Data
GROUP BY exptId;
\end{lstlisting}

\begin{lstlisting}[style=output]
+--------+------------+
| exptId | AVG(level) |
+--------+------------+
| 1      |   125.3333 |
| 2      |    95.3333 |
| 3      |   126.3333 |
| 4      |    83.5000 |
| 5      |    92.7500 |
| 6      |    18.3333 |
| 7      |    20.0000 |
| 8      |    40.0000 |
| 9      |    20.0000 |
+--------+------------+
9 rows in set (0.00 sec)
\end{lstlisting}

%--------------------
\subsection*{Question 11}
What is the average expression level of each array probe (\inline{affyId}) across all experiments?

\subsubsection*{Answer}

\begin{lstlisting}[language=sql]
SELECT affyId, AVG(level) FROM Data
GROUP BY affyId;
\end{lstlisting}

\begin{lstlisting}[style=output]
+---------------------------+------------+
| affyId                    | AVG(level) |
+---------------------------+------------+
| 31315_at                  |   250.0000 |
| 31324_at                  |    91.0000 |
| 31325_at                  |    89.0000 |
| 31356_at                  |    91.0000 |
| 31362_at                  |   260.0000 |
| 31510_s_at                |   257.0000 |
| 5321_at                   |    90.0000 |
| 5322_at                   |    90.0000 |
| 5323_at                   |    90.0000 |
| 5324_at                   |    73.5000 |
| 5325_at                   |    90.0000 |
| AFFX-BioB-3_at            |    97.0000 |
| AFFX-BioB-5_at            |    20.0000 |
| AFFX-BioB-M_at            |    62.8000 |
| AFFX-HSAC07/X00351_M_at   |    86.0000 |
| AFFX-HUMBAPDH/M33197_3_st |   277.0000 |
| AFFX-HUMTFFR/M11507_at    |    90.0000 |
| AFFX-M27830_3_at          |   271.0000 |
| AFFX-MurIL10_at           |     6.6667 |
| AFFX-MurIL2_at            |    20.0000 |
| AFFX-MurIL4_at            |    49.0000 |
| U95-32123_at              |   128.0000 |
| U98-40474_at              |    57.0000 |
+---------------------------+------------+
23 rows in set (0.00 sec)
\end{lstlisting}


%--------------------
\subsection*{Question 12}
What is the purpose of the following query?

\begin{lstlisting}[language=sql]
SELECT Data.affyId, Data.level, Data.exptId, DataCopy.affyId, DataCopy.level, DataCopy.exptId
FROM Data, Data DataCopy
WHERE Data.level > 10 * DataCopy.level AND
      Data.affyId=DataCopy.affyId AND
      Data.affyId LIKE "AFFX%"
LIMIT 10;
\end{lstlisting}

\subsubsection*{Answer}

\begin{lstlisting}[style=output]
+----------------+-------+--------+----------------+-------+--------+
| affyId         | level | exptId | affyId         | level | exptId |
+----------------+-------+--------+----------------+-------+--------+
| AFFX-BioB-M_at |   214 | 5      | AFFX-BioB-M_at |    20 | 3      |
| AFFX-BioB-M_at |   214 | 5      | AFFX-BioB-M_at |    20 | 7      |
| AFFX-BioB-M_at |   214 | 5      | AFFX-BioB-M_at |    20 | 9      |
+----------------+-------+--------+----------------+-------+--------+
3 rows in set (0.00 sec)
\end{lstlisting}

What this query is doing is to make a copy of the \inline{Data} table in order to compare the \inline{affyId}s that start with \inline{AFFX} to show the pairs that fulfil the condition of having an expression \texttt{level} of one being 10 times larger than the other one. The query lets us not only compare the expression \texttt{level}, but the \inline{exptId} value as well.

% More specifically, the query selects the columns \inline{affyId}, \texttt{level} and \inline{exptId} from \inline{Data} and also the same ones from \inline{DataCopy}, which is an exact copy of table \inline{Data}. This copy has been done in the \inline{FROM} clause.

% In the \inline{WHERE} clause, 3 conditions are introduced:
% \begin{enumerate}[(i)]
% 	\item Select those rows in \texttt{level} from\inline{Data} table which are more than 10 times higher than the rows in \texttt{level} from the copy of\inline{Data} table.
% 	\item Select those rows in which \inline{affyId} from tables\inline{Data} and it's copy coincide.
% 	\item Take those entries in \inline{affyId} from\inline{Data} that start with \inline{AFFX}.
% \end{enumerate}

The \inline{LIMIT 10} clause is there to show only the 10 first results, although it is unnecessary for this specific query, as there are only 3 results.

%--------------------
\subsection*{Question 13}
Write a query to provide three different descriptions for all \inline{gbId} in table \inline{Targets}.

\subsubsection*{Answer}

The tricky part with creating this query is that we have 4 different descriptions (in tables \inline{Descriptions}, \inline{LocusDescr}, \inline{UniDescr}, and \inline{GO_Descr}), so depending on which ones we choose, we will have different results.

Apart from that, one has to be careful on using the right relational path to connect all the tables properly. We will show two queries: (a) one to show \inline{Descriptions.description}, \inline{LocusDescr.description}, and \inline{UniDescr.description}; and (b) another one to show \inline{Descriptions.description}, \inline{GO_Descr.description}, \inline{LocusDescr.description}.

For space reasons, we will only show a sample of the output.

\begin{lstlisting}[language=sql]
SELECT Targets.gbId, Descriptions.description AS description, LocusDescr.description AS LocusDescr, UniDescr.description AS UniDescr
FROM Targets, Descriptions, LocusDescr, UniDescr, UniSeqs
WHERE Targets.gbId=Descriptions.gbId AND
      Targets.species=LocusDescr.species AND
      UniSeqs.uId=UniDescr.uId AND
      Targets.gbId=UniSeqs.gbId
ORDER BY gbId;
\end{lstlisting}

\begin{lstlisting}[style=output]
+--------+-------------+------------+----------------+
| gbId   | description | LocusDescr | UniDescr       |
+--------+-------------+------------+----------------+
| S75295 | Glucan      | Collagen   | Glucan         |
| S75295 | Glucan      | serine     | Glucan         |
| S75295 | Glucan      | Glucan     | Glucan         |
| S75295 | Glucan      | albumin    | Glucan         |
| S75295 | Glucan      | granulysi  | Glucan         |
| ...    | ...         | ...        | ....           |
+--------+-------------+------------+----------------+
42 rows in set (0.00 sec)
\end{lstlisting}

\begin{lstlisting}[language=sql]
SELECT Targets.gbId, Descriptions.description AS description, GO_Descr.description AS GO_Descr, LocusDescr.description AS LocusDescr
FROM Targets, Descriptions, GO_Descr, LocusDescr, Ontologies, LocusLinks
WHERE Descriptions.gbId=Targets.gbId AND
      GO_Descr.goAcc=Ontologies.goAcc AND
      LocusDescr.linkId=Ontologies.linkId AND
      Targets.gbId=LocusLinks.gbId
ORDER BY gbId;
\end{lstlisting}

\begin{lstlisting}[style=output]
+--------+-----------------+--------------+------------+
| gbId   | description     | GoDescr      | LocusDescr |
+--------+-----------------+--------------+------------+
| A00142 | granulysin      | Glucan Enz   | Glucan     |
| A00142 | granulysin      | Glucan Enz   | Glucan     |
| A00142 | granulysin      | Serine Prot. | Collagen   |
| A00142 | granulysin      | Serine Prot. | Collagen   |
| A00146 | lypase, gastric | Glucan Enz   | Glucan     |
| ...    | ...             | ...          | ....       |
+--------+-----------------+--------------+------------+
40 rows in set (0.00 sec)
\end{lstlisting}

% \begin{lstlisting}[language=sql]
% SELECT Targets.gbId, Descriptions.description AS description, GO_Descr.description AS GO_Descr, LocusDescr.description AS LocusDescr
% FROM Targets, Descriptions, Ontologies, GO_Descr, LocusLinks, LocusDescr
% WHERE Descriptions.gbId=Targets.gbId AND
%       Targets.gbId=LocusLinks.gbId AND
%       LocusLinks.linkId=LocusDescr.linkId AND
%       LocusLinks.linkId=Ontologies.linkId AND
%       Ontologies.goAcc=GO_Descr.goAcc
% ORDER BY gbId;
% \end{lstlisting}

% \begin{lstlisting}[style=output]
% +--------+-------------+--------------+------------+
% | gbId   | description | GO_Descr     | LocusDescr |
% +--------+-------------+--------------+------------+
% | L02870 | kinase      | Serine Prot. | Collagen   |
% | S75295 | Glucan      | Glucan Enz   | Glucan     |
% +--------+-------------+--------------+------------+
% 2 rows in set (0.00 sec)
% \end{lstlisting}

%--------------------
\subsection*{Question 14}
Write a query to provide all gene ontology (\inline{GO_descr}) descriptions related with all  species in table \inline{Targets} sorted alphabetically and providing the first five results. Export the query to a tab-separated-file with the command:

\begin{lstlisting}[language=sql]
SELECT * FROM TABLE INTO OUTFILE 'data.out';
\end{lstlisting}

\subsubsection*{Answer}
We just need to do the usual query and add the \inline{INTO OUTFILE} expression at the end of it, as it can be seen below.
\begin{lstlisting}[language=sql]
SELECT Targets.species, GO_Descr.description
FROM Targets, GO_Descr, Ontologies, LocusDescr
WHERE Targets.species=LocusDescr.species AND
      LocusDescr.linkId=Ontologies.linkId AND
      Ontologies.goAcc=GO_Descr.goAcc
ORDER BY species
INTO OUTFILE 'data.out';
\end{lstlisting}

\begin{lstlisting}[style=output]
+---------+--------------+
| species | description  |
+---------+--------------+
| Hs      | Serine Prot. |
| Hs      | Serine Prot. |
| Hs      | Serine Prot. |
| Hs      | Serine Prot. |
| Hs      | Serine Prot. |
+---------+--------------+
5 rows in set (0.00 sec)
\end{lstlisting}



%-----------------------------------------------------------------
%	BIBLIOGRAPHY
%-----------------------------------------------------------------
% \nocite{o:genome}

\printbibliography[heading=bibintoc]
% \setcounter{secnumdepth}{0}
% \section{References}
% \printbibliography[title={Articles}, type=article, heading=subbibliography]
% \printbibliography[title={Books}, type=book, heading=subbibliography]
% \printbibliography[title={Websites}, type=online , heading=subbibliography]
% \printbibliography[title={Other}, type=misc , heading=subbibliography]

\end{document}
